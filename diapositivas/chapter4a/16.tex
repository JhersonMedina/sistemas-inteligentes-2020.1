\definecolor{Red}{RGB}{150,0,0}
\definecolor{Purple}{RGB}{166,115,255}
\definecolor{Green}{RGB}{0,100,0}
\definecolor{Blue}{RGB}{0,12,230}

    % Pagina 16%
    \begin{frame}{Búsqueda A*}
        \small{
            \textcolor{Green}{Idea}: evitar expandir caminos que de por si ya son costosos.
            \break\break\
            Funcion de evaluacion \textcolor{Purple}{$f(n)=g(n)+h(n)$}.
            \break\break
            \textcolor{Purple}{$g(n)$} = costo acumulado para llegar a \textcolor{Purple}{$n$} \\
            \textcolor{Purple}{$h(n)$} = costo estimado a la meta desde \textcolor{Purple}{$n$} \\
            \textcolor{Purple}{$f(n)$} = costo estimado de la ruta a la meta, a través de      \textcolor{Purple}{$n$} 
            \break\break
            La búsqueda A* utiliza heurísticas \textbf{\textcolor{Blue}{admisibles}}\\
            es decir, \textcolor{Purple}{$h(n) \leq h^*(n)$}, donde \textcolor{Purple}{$h^*(n)$} es el costo \textbf{\textcolor{Red}{real}} de \textcolor{Purple}{$n$}.\\
            (Tambien requiere \textcolor{Purple}{$h(n) \geq 0$}, entonces \textcolor{Purple}{$h(G)=0$} para cualquier meta \textcolor{Purple}{$G$})
            \break\break
            Por ejemplo, \textcolor{Purple}{$h_{SLD}(n)$} nunca sobreestima la distancia real del camino
            \break\break
            \textbf{\textcolor{Green}{Teorema}}: La búsqueda A* es óptima
        }\break\break\break
    \end{frame}
